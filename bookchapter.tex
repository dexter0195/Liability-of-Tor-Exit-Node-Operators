
%%%%%%%%%%%%%%%%%%%%%%% file typeinst.tex %%%%%%%%%%%%%%%%%%%%%%%%%
%
% This is the LaTeX source for the instructions to authors using
% the LaTeX document class 'llncs.cls' for contributions to
% the Lecture Notes in Computer Sciences series.
% http://www.springer.com/lncs       Springer Heidelberg 2006/05/04
%
% It may be used as a template for your own input - copy it
% to a new file with a new name and use it as the basis
% for your article.
%
% NB: the document class 'llncs' has its own and detailed documentation, see
% ftp://ftp.springer.de/data/pubftp/pub/tex/latex/llncs/latex2e/llncsdoc.pdf
%
%%%%%%%%%%%%%%%%%%%%%%%%%%%%%%%%%%%%%%%%%%%%%%%%%%%%%%%%%%%%%%%%%%%


\documentclass[runningheads,a4paper]{llncs}

\usepackage{amssymb}
\setcounter{tocdepth}{3}
\usepackage{graphicx}

\usepackage{url}
\urldef{\mailsa}\path|{alfred.hofmann, ursula.barth, ingrid.haas, frank.holzwarth,|
\urldef{\mailsb}\path|anna.kramer, leonie.kunz, christine.reiss, nicole.sator,|
\urldef{\mailsc}\path|erika.siebert-cole, peter.strasser, lncs}@springer.com|    
\newcommand{\keywords}[1]{\par\addvspace\baselineskip
\noindent\keywordname\enspace\ignorespaces#1}

\begin{document}

\mainmatter  % start of an individual contribution

% first the title is needed
\title{On the liability of TOR Exit Relay operators}

% a short form should be given in case it is too long for the running head
\titlerunning{}

% the name(s) of the author(s) follow(s) next
%
% NB: Chinese authors should write their first names(s) in front of
% their surnames. This ensures that the names appear correctly in
% the running heads and the author index.
%
\author{Alessio Trivisonno}
%
\authorrunning{}
% (feature abused for this document to repeat the title also on left hand pages)

% the affiliations are given next; don't give your e-mail address
% unless you accept that it will be published
\institute{University of Trento}

%
% NB: a more complex sample for affiliations and the mapping to the
% corresponding authors can be found in the file "llncs.dem"
% (search for the string "\mainmatter" where a contribution starts).
% "llncs.dem" accompanies the document class "llncs.cls".
%

\toctitle{}
\tocauthor{}
\maketitle


\begin{abstract}
The abstract should summarize the contents of the paper and should
contain at least 70 and at most 150 words. It should be written using the
\emph{abstract} environment.
\keywords{We would like to encourage you to list your keywords within
the abstract section}
\end{abstract}


\section{Introduction}

\section{Background}

\subsection{What is TOR}

Tor is a free and open source software aimed to enable 
anonymous communication on the web. It is one the most used and secure \textit{Privacy Enhancing Technology} that is available now.\footnote{Although the technology is very strong and secure there are still many attacks that can undermine the anonymity of the Tor users} We can find it in the same class as VPNs, Proxies and 
DNS Bypass. However it is the only tecnique that can truly ensure your anonymity on the web. For this reason Tor has become more and more popular during these years, expecially in countries where censorsihp is most suffered. Today Tor is used on average by 2 Million users every day.

The name Tor stands for ``The Onion Router`` 
which is the technology at the core of this software.


The Onion Routing is a networking mechanism that ensures encryption and 
anonymity in a communication between two endpoints. The connection between 
endpoints is routed along an encrypted chain where every node in the chain is 
called a relay. Each relay has the knowledge of only the previous and the 
following nodes in the chain. This makes the complete picture of the communication hidden from anyone. \cite{CCDCOF}


\subsection{How It Works}
The Tor architecture is composed of basically three entities: 
\begin{itemize}
    \setlength\itemsep{1em}
    \item The Directory Servers
    \item Middle Relays
    \item Exit Relays
\end{itemize}
The Directory Servers advertise the entry points of the Tor Network to the client while the network
itself if composed by Middle Relays and Exit Relays. 
Tor creates a circuit starting from the end user to the destination that jumps on 
different hops during its path and at each jump the intermediate packet is encapsulated 
into an additional layer of encryption. From the moment that the request of the end user 
enters the network is jumps randomly on a set of Middle Relays before reaching the Exit Relay. 
Exit relays are the points in which the request of the end user makes the last step towards 
the final destination. See Fig \ref{fig:fig_tor_arch}

\begin{figure}[]
    \center{\includegraphics[width=0.5\textwidth]
        {./figures/tor_arch.png}}
        \caption{ Tor architecture}
        \label{fig:fig_tor_arch}
\end{figure}

All the relays in the network are run by individual volunteers or organizations and the technology 
relies on this to guarantee the anonymity.

As shown in Fig \ref{fig:fig_tor_arch} the connection between client/relay and relay/relay is encrypted
and does not cause any problems to the operator. The Exit Relays insted, are the one that carry out the 
last-mile communication. They are the most exposed part of the network, since the comminication between
the client and the server is at this stage unencrypted, making the request look like as it is the 
Exit Relay which is originating that traffic. 

\section{For what Tor is used for}
Tor has been developed with the idea that anyone should be able to express their belief without beeing monitored by others, not even governments. Among its main usage we can find privacy protection for individual that are unwilling to give away their navigation data to ISPs. In the business filed Tor is used to ensure confidentiality and for doing research on competitors without leaving traces. Similarly Tor can be used for intelligence gathering by government's body like U.S. Navy, in fact the very technology of Onion Routing was developed by them. It can also be used to protect the identity of sources for journalists, like whistleblowers. And It can also help activists to report abuse or organize meetings, as it happened during the Egyptian Revolution of 2011. \cite{WASH_715}

\begin{figure}[]
    \center{\includegraphics[width=0.5\textwidth]
        {./figures/tor_illicit.png}}
        \caption{Web-Based Onion Services in February 2016 (Source Wikipedia.org)}
        \label{fig:fig_tor_ill}
\end{figure}

Beside that Tor has been accused to be full with criminal activity, especially regarding child pornography and drug dealing. As shown in figure \ref{fig:fig_tor_ill} the amount of illicit activity percentage on the overall onion services is high (almost 30\%). For example famous were the cases of the sites PlayPen and SilkRoad that led to the incarceration of many users and maintainer of those sites. 

\section{The problem: Liability of Tor Exit Relays Operators}
Since the Tor Exit Relay Operators are the one that carry on the \textit{last mile communication} they are the one that are most exposed in terms of legal actions. In fact if from their server transit a request from a generic Tor user for a site that hosts child pornography, such request cannot be distinguished from the normal traffic of the operator. 
Exit Relays behave in the same way normal relay does, they remove the source ip address and replaces it with its own ip address as a source, lastly it forwards the request to the next hop. In this case the last hop is the final destination and as shown in Fig. \ref{fig:fig_tor_arch} the communication between the Exit Relay and the destination server is no more encrypted by the Tor protocol \footnote{The communication can still be encrypted at application level with HTTPS but the source and destination ip address are no longer anonymised}. 
TODO: Company Terms of Service, like Aruba

\section{The case Austrian Regional Crime Court vs William Weber}
In this section I will present the case of Austrian regional criminal court in Graz vs. W. Weber. This case sees W.Weber, a tor exit node operator being sentenced by the court for aiding the distribution of child pornography. I will discuss the case and talk about the reason why he was held accountable and what where the court motivation for that.
asdasd

\section{Liability of Internet Service Providers for Copyright Infringement}
In this section I will present a similar scenario but in the case of copyright infringement. I will discuss the mere conduit safe harbor that can be applied to ISP in case of copyright infringement correlating the it with court decisions. I will also discuss wether a tor operator can be considered an ISP. This in correlation with the Directive 2000/31/EC of the European Parliament and of the Council of 8 June 2000.

\section{The situation in Italy}
In this section I will present how in Italy is handled the liability of ISP illustrating some examples on court decision, as the sentence n. 54946/2016 of the Corte di Cassazione. And also I will discuss the d.lgs 70/2003, with particular interest to art 14, art 15 and art 16.

\section{Conclusion}
\blindtext[3]



\begin{thebibliography}{9}

\bibitem{CCDCOF}
Emin \c{C}alı\c{s}kan, Tom\'{a}\u{s} Min\'{a}rik, Anna-Maria Osula (2015).
\textit{Technical and Legal Overview of the 
Tor Anonymity Network}.\\
NATO Cooperative Cyber Defence Center of 
Excellence, Tallin, Estonia

\bibitem{WASH_715}
    Keith D. Watson. (2012)
    \textit{ THE TOR NETWORK: A GLOBAL INQUIRY INTO THE LEGAL STATUS OF ANONYMITY NETWORKS . }
    Washington University Global Studies Law Review

\bibitem{EFF_PLAYPEN}
    Electronic Frontier Foundation\\
    \textit{The Playpen Cases: Frequently Asked Questions}
    link: https://www.eff.org/it/pages/playpen-cases-frequently-asked-questions\\
    date of last visit: "03/04/2019"

\end{thebibliography}


\end{document}